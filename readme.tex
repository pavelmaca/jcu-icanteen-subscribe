\documentclass[a4]{article}
\usepackage{xltxtra,polyglossia} 
\setdefaultlanguage{czech} 
\usepackage[]{graphicx}
\usepackage{listings}
\usepackage{color}
\usepackage{blindtext}
\usepackage{scrextend}
\addtokomafont{labelinglabel}{\sffamily}

\definecolor{codegreen}{rgb}{0,0.6,0}
\definecolor{codegray}{rgb}{0.5,0.5,0.5}
\definecolor{codepurple}{rgb}{0.58,0,0.82}
\definecolor{backcolour}{rgb}{0.95,0.95,0.92}

\lstdefinestyle{mystyle}{
    backgroundcolor=\color{backcolour},
    commentstyle=\color{codegreen},
    keywordstyle=\color{magenta},
    numberstyle=\tiny\color{codegray},
    stringstyle=\color{codepurple},
    basicstyle=\footnotesize,
    breakatwhitespace=false,
    breaklines=true,
    captionpos=b,
    keepspaces=true,
    numbers=left,
    numbersep=5pt,
    showspaces=false,
    showstringspaces=false,
    showtabs=false,
    tabsize=2
}

\lstset{style=mystyle}

\title{Odběr jídelníčků z menzy JCU}
\author{Pavel Máca}

\begin{document}
\begin{titlepage}
	\centering
	{\Huge Operační systémy 2\par}
	\vspace{1cm}
	{\Large Zimní semestr 2016/17\par}
	\vspace{1.5cm}
	{\huge\bfseries Odběr jídelníčků z menzy JČU\par}
	\vspace{2cm}
	{\Large Pavel Máca\par}
	\vfill


	{\large \today\par}
\end{titlepage}

\pagebreak

\section{Úvod}
Účelem projektu se vytvořit emailové notifikace na nově vystavený jídelní lístek v menze jihočeské Univerzity.

Uživatel má možnost s přihlásit k odběru, který funguje na principu newsletteru. Na stránkách aplikace pouze vyplní svůj email. Při dalším vystavení jídelníčku ho bude aplikace informovat o specialitách, které si může objednat.

Mnohokrát se stává, že si chce někdo objednat specialitu přes internetové stránky menzy, ale ta je již vyprodaná. Díky včasnému upozornění lze vytvořit objednávku vždy v čas.


\section{Metodika}
Jako programovací jazyk bylo zvoleno PHP. Kód může běžet jak na Windows, tak OS Linux  a dalších OS, pro které existuje zkompilované PHP.

\section{Popis skriptu}
\subsection{Konfigurace}
Před spuštěním je potřeba vytvořit soubor s konfigurací
\verb@config.local.neon@

Do něj vložte konfiguraci SMTP připojení pro server odesílající poštu.
\begin{lstlisting}
parameters:
    mail:
          host: smtp.gmail.com
          username: ***user***
          password: ***pass****
          secure: ssl
\end{lstlisting}

\subsection{Konzole}
Aplikaci lze spustit pouze z příkazového řádku. Před jejím spuštěním je zapotřebí doinstalovat veškeré závislosti pomocí programu Composer\footnote{https://getcomposer.org - balíčkovací systém PHP}.


Inicializace závislostí PHP se provádí příkazem \verb@composer install@


Aplikaci lze ovládat z příkazové řádky pomocí příkazu ve tvaru \\
\verb@php www/index.php <přikaz> [<argument>]}@

\vspace{3mm}
\textbf{Seznam dostupných příkazů} \\
\begin{tabular}{ l l }
	\verb@subscribe <email>@ & Přidá email do seznamu odběratelů \\
	\verb@unsubscribe <emai>@ &  Odebere email ze seznamu odběratelů \\
	\verb@notification-send@  & Odešle notifikace, pokud existuje nový jídelníček\\
	\verb@subsciption-list@   & Vypíše seznam odběratelů\\
\end{tabular}



\subsection{Webové rozhraní - demo}
Demo běžící aplikace je k dispozici na adrese https://menza-jcu.assassik.cz
Kontrola jídelníčku probíhá v hodinových intervalech pomocí nastaveného cronu a příkazu \verb@notification-send@.

\subsection{Funkčnost}
Script načte aktuálně zveřejněný jídelníček, zpracuje HTML kód ve kterém se nachází data a názvy jídel.

Datum posledního jídla je uloženo do paměti scriptu zvaná „cache“, podle které se rozpoznávají nová jídla při dalším spuštění.

V okamžiku, kdy se uživatel přihlásí je jeho email zapsán do SQLite databáze. Při nalezení nového jídla script automaticky pomocí SMTP protokolu rozesílá jednotlivé upozornění na všechny emaily, která se nacházejí v databázi.

Pokud si uživatel nepřeje dostávat další upozornění, nachází se v emailu odkaz pro odhlášení z odběru.

Jelikož je jídelníček zveřejňován třetí stranou, nelze zajistit správný chod aplikace po delší dobu, pokud třetí strana nějak změní datovou strukturu jídelníčku.

\section{Zdrojový kód}
Zdrojová kód je k dispozici na adrese https://github.com/pavelmaca/jcu-icanteen-subscribe

Vybraná část kódu:
\lstset{language=PHP}
\begin{lstlisting}
$str = $td->textContent
switch ($colCount) {
  case 0:
    // fisr col contains date
    if (preg_match('~^([0-9]{1,2}\.){2}[0-9]{4}$~', $str)) {
      $date = DateTime::createFromFormat('j.n.Y H:i:s', $str . ' 00:00:00');
    }
    break;
  case 1:
    // type of meal
    $str = self::fixEncoding($str);
    if (preg_match('~^(Specialita) [0-9]~', $str)) {
      $type = $str;
    } else {
      $skip = true;
    }
    break;
  case 3:
    // name of a meal
    $name = $str;
    break;
  }
  $colCount++;
}
\end{lstlisting}

\section{Závěr}
Aplikace momentálně podporuje pouze upozornění na speciality z jedné stravovny. Do budoucna by se mohla rozšířit o další funkce, jako výběr typů jídel a další provozovny.

\end{document}